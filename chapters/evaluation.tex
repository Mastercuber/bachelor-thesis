%% LaTeX2e class for student theses
%% sections/evaluation.tex
%%
%% Karlsruhe University of Applied Sciences
%% Faculty of  Computer Science and Business Information Systems
%% Distributed Systems (vsys)
%%
%% Prof. Dr. Christian Zirpins
%% christian.zirpins@hs-karlsruhe.de
%%
%%
%% Version 0.2, 2017-11-15
%%
%% --------------------------------------------------------
%% | Derived from sdqthesis by Erik Burger burger@kit.edu |
%% --------------------------------------------------------

\chapter{Evaluation}
\label{ch:Evaluation}

\section{Anwendungsbeispiel}
\section{(Performanzmessungen)}
\begin{tabularx}{\textwidth}{p{0.15\textwidth}|r|X|X|r}
	 & rsa-md4 & rsa-md5 & rsa-sha256 & rsa-sha512\\
	\hline
	Testlauf 1& 107,177 ms& 128.993 ms& 121.406 ms& 114.669 ms\\
	Testlauf 2& 43,660 ms& 51.178 ms& 35.524 ms& 39.633 ms\\
	Testlauf 3& 32.101 ms& 32.691 ms& 30.461 ms& 32.293 ms\\
	Testlauf 4& 36.096 ms& 30.279 ms& 32.018 ms& 43.842 ms\\
	Testlauf 5& 31.869 ms& 34.882 ms& 30.877 ms& 26.555 ms\\
	Testlauf 6& 39.218 ms& 67.507 ms& 38.734 ms& 31.330 ms\\
	Testlauf 7& 40.996 ms& 39.646 ms& 45.433 ms& 27.290 ms\\
	Testlauf 8& 25.977 ms& 39.302 ms& 26.778 ms& 53.478 ms\\
	Testlauf 9& 23.641 ms& 37.833 ms& 30.485 ms& 36.284 ms\\
	Testlauf 10& 31.097 ms& 52.725 ms& 28.943 ms& 28.634 ms\\
	\hline
	Arithmetisches Mittel& 41,183 ms& 51,503 ms& 42,065 ms& 43,400 ms\\
\end{tabularx}
\section{Disskusion von Vor- und Nachteilen der Lösung}
