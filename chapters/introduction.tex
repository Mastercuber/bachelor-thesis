%% LaTeX2e class for student theses
%% sections/content.tex
%%
%% Karlsruhe University of Applied Sciences
%% Faculty of  Computer Science and Business Information Systems
%% Distributed Systems (vsys)
%%
%% Prof. Dr. Christian Zirpins
%% christian.zirpins@hs-karlsruhe.de
%%
%%
%% Version 0.2, 2017-11-15
%%
%% --------------------------------------------------------
%% | Derived from sdqthesis by Erik Burger burger@kit.edu |
%% --------------------------------------------------------
\chapter{ 
	\iflanguage{english}{Introduction}{Einleitung}
}
\label{ch:Introduction}
\iflanguage{english}{
	\todo{- Justification of this work}
	Centralization of data and trust in individual instances is ubiquitous today. On the internet there are lots of social networks like Facebook, Twitter, Google+, Instagram or Pinterest which mostly keep the right to the data you get from their users and last but not least sell these data for e.g. advertising purposes. It is also easier for crackers to access the entire database, since centralized networks usually also have a central database or a central access point. When the access to this/these is reached, the whole database can be read out.\\
	
	What does decentralized mean? In politics, decentralization \glqq means the transfer of central state tasks to subnational or subsidiary level(s) \grqq\cite{wikipedia-decentralization-politics}. In the energy industry, one speaks of a \glqq decentralized power generation\grqq if the power is also generated at the places where it is consumed. An example of this would be a hydroelectric power plant that supplies electricity to the surrounding villages or cities\cite{wikipedia-decentralization-energy}. Decentralization in computer science means the meshed networking of computers with each other\cite{wikipedia-decentralization-informatics}.\\
	
	The allocation of central government tasks to sub-national or subsidiary levels relieves the burden on the central government level and thus has more resources for other tasks. When hydropower plants are built close to villages and towns, there is no need to transport electricity, which reduces the loss of electricity when it is transported via the power grid. The decentralisation of social networks has several advantages. For example, it becomes possible to moderate the individual instances themselves and thus decentralise control or create content from one social network to another. With ActivityPub this is guaranteed by the promotion.\\
	
	ActivityPub consists of two parts. The client-to-server protocol and the promoted server-to-server protocol. The ActivityPub W3C recommendation in early 2018 and the Social-Web-Working-Group of the W3C, as well as others, will advance the dissemination of the protocol, probably to take advantage of the network effect. The network effect is a term derived from economics. In terms of translation it means: "As soon as the critical mass of networks has been reached, the positive feedback effect occurs, so that the benefit for networks implementing the standard increases".\\
	
	\todo{- Purpose of this work}
	In order to complete the work successfully, an implementation of the distributed server-to-server protocol as a prototype and possibly a concept for the secure use of the protocol should be available. In addition, the work should give an overview of the basics of the protocol and how the relevant components are to be understood. In this context \glqq secure\grqq means secure authentication of the client against the server and additionally the server among each other, as well as ensuring the manipulation-free transmission of the contents.\\
	
	\todo{-	Delineation of the topic and topic related definitions}
	
	\todo{- History and level of research}
	The ActivityPub\cite{activityPub} standard was recommended by the W3C on 23 January 2018 and developed by a working group of the W3C, the Social Web Working Group (SWWG)\cite{socialWg,pushSocialWeb}. This group was active from July 21, 2014 to February 13, 2018\cite{socialWg} and developed among others ActivityPub, Activity Streams Core\cite{activityStreamsCore}, Activity Streams Vocab\cite{activityStreamsVocabulary}. The aim of the SWWG was to define new protocols, vocabularies and API's for the access to social contents of the so-called Open Web Platform\cite{social-wg-charter}.

}{
	%	- 	Rechtfertigung der Arbeit
	Zentralisierung von Daten und das Vertrauen auf einzelne Instanzen ist heutzutage allgegenwärtig. Im Internet gibt es eine Fülle an sozialen Netzwerken wie Facebook, Twitter, Google+, Instagram oder Pinterest die zumeist das Recht an den Daten, die Sie von ihren Nutzer bekommen, behalten und nicht zuletzt diese Daten auch verkaufen für z. B. Werbezwecke. Für Kriminelle sowie Geheimdienste ist es außerdem leichter an die gesamte Datenbank zu kommen, da bei zentralisierten Netzwerken meist auch eine zentrale Datenbank, bzw. ein zentraler Zugriffspunkt, vorhanden ist. Wenn der Zugriff auf diesen zentralen Punkt erreicht ist kann die ganze Datenbank ausgelesen werden.\\
	
	Was bedeutet nun dezentral? In der Politik wird unter Dezentralisierung \glqq die Übertragung zentral staatlicher Aufgaben auf subnationale oder subsidiäre Ebene(n) verstanden\grqq\cite{wikipedia-dezentralisierung-politik}. In der Energiewirtschaft spricht man von einer \glqq dezentralen Stromerzeugung\grqq, wenn der Strom an den Stellen wo er verbraucht auch erzeugt wird. Ein Beispiel hierfür wäre ein Wasserkraftwerk, dass den Strom für die umgebenen Dörfer oder Städte liefert\cite{wikipedia-dezentralisierung-energie}. In der Informatik versteht man unter Dezentralisierung das verteilen von u. A. Daten über mehrere unabhängige Server hinweg.\\
	
	Durch die Verteilung der Aufgaben auf sub-nationale oder subsidiäre Ebene wird die zentral staatliche Ebene entlastet und hat somit mehr Ressourcen für andere Aufgaben. Beim Errichten von Wasserkraftwerken nahe an Dörfern und Städten entfällt der Transport des Stroms über größere Distanzen und somit verringert sich der Verlust beim Transport über das Stromnetz. Dezentralisieren von sozialen Netzwerken bring verschiedene Vorteile mit sich. Ein Vorteil ist die Skalierbarkeit bei dezentralen sozialen Netzwerken. Durch das hinzufügen weiterer Instanzen können dem Netzwerk neue Ressourcen bereitgestellt werden. Ein weiterer Vorteil liegt darin, dass keine zentrale Kontrollinstanz vorhanden ist welche korrumpiert werden kann, sei es durch Kriminelle, wirtschaftliche Interessenvertreter oder staatliche Autoritäten.\\
	
	%	- 	Ziel der Arbeit
	Ziel der Arbeit ist es die Implementierung des verteilten Server-zu-Server Protokolls als Prototyp vorzunehmen und die bis dato von der \gls{w3c} Gemeinschaft empfohlenen Sicherheitsstandards für das ActivityPub Protokoll zu prüfen. Darüber hinaus soll die Arbeit einen Überblick über die Grundlagen des Protokolls geben und wie die relevanten Bestandteile zu verstehen sind. Die verwendeten Sicherheitsstandards beziehen sich auf das Authentifizieren vom Client gegenüber dem Server und zusätzlich der Server untereinander, sowie das Sicherstellen der manipulationsfreien Übertragung von Inhalten.\\
	\todo{@Zirpins Was genau soll im oberen Absatz weiter erklärt werden???}
	% Schutzziele
	%	- 	Abgrenzung des Themas und Themenbezogenen Definitionen
		
	%	- 	Geschichte und Stand der Forschung:
	
	%Der ActivityPub Standard wurde am 23 Januar 2018 von der \gls{w3c} %empfohlen\cite{activityPub} und von einer Arbeitsgruppe des \gls{w3c}, der %\gls{swwg}\cite{socialWg,pushSocialWeb}, entwickelt. Diese Gruppe war vom 21. Juli 2014 %bis zum 13 Februar 2018 aktiv\cite{socialWg} und entwickelte unter anderem ActivityPub, %\gls{asc}\cite{activityStreamsCore} und \gls{asv}\cite{activityStreamsVocabulary}. Die %\gls{swwg} war eine Arbeitsgruppe des \gls{w3c} mit dem Ziel neue Protokolle, Vokabulare %und API's zu definieren für den Zugriff auf soziale Inhalte der sogenannten %\gls{owp}\cite{social-wg-charter}.\\
}

\section{Motivation}
\label{sec:Introduction:Motivation}
\iflanguage{english}{
	With central social networks, e.g. Facebook, the company has control over the data. Through decentralization, the data remains with the individual instances of the social network. The databases in central networks can be distributed and the network can also be decentralized, but the data remains with the company. In addition, centralized social networks cannot easily share content with other social networks. With a protocol such as ActivityPub, this is distributed and access to content and exchange of activities across multiple social networks is possible.\\
	
	Protocols such as \glqq Diaspora Federation and OStatus\grqq~provide access to decentralized social content and enable the publication of social content. The standard recommended in March last year by the W3C called \glqq ActivityPub\grqq~ was developed on the basis of knowledge in dealing with the OStatus and Pump.io protocol\cite{activityPub-acknowledgements}. ActivityPub is like \glqq OStatus\grqq~ a protocol for decentralized social networks and is examined in this Bachelor thesis.
}{
	Bei zentralen sozialen Netzwerken, z.B. Facebook, besitzt die Kontrolle über Daten das Unternehmen. Durch die Dezentralisierung bleiben Daten bei den einzelnen Instanzen des sozialen Netzwerkes. Die Datenbanken bei zentralen Netzwerken können zwar verteilt sein und auch das Netzwerk könnte dezentral angelegt sein, trotzdem gehören die Daten einem Unternehmen. Außerdem können bei zentralen sozialen Netzwerken nicht ohne weiteres Inhalte mit anderen Netzwerken ausgetauscht werden. Mit einem Protokoll wie ActivityPub wird das Verteilen sowie der Zugriff auf Inhalte und der Austausch von Aktivitäten über mehrere soziale Netzwerke hinweg ermöglicht.\\
	
	ActivityPub besteht aus zwei Teilen. Dem Client-zu-Server Protokoll, auch \glqq Social API\grqq~genannt, und dem förderierten Server-zu-Server Protokoll. Durch die ActivityPub \gls{w3c} Empfehlung Anfang 2018 und die \gls{swwg} Arbeitsgruppe des \gls{w3c} sowie weiteren, wird die Verbreitung des Protokolls vorangebracht.\\
	
	Der Netzwerkeffekt ist ein aus der Volkswirtschaftslehre stammender Begriff. Als Beispiel sei das Telefonnetz genannt. Umso mehr Leute ein Telefon besitzen, umso höher ist der Nutzen für jeden einzelnen Telefon Besitzer\cite{netzwerkeffekt}. Übertragen auf ActivityPub bedeutet das soviel wie: \glqq Je mehr Netzwerke den Standard implementieren, desto höher ist der Nutzen für ein einzelnes Netzwerk und im Endeffekt für die einzelnen Nutzer\grqq.\\
	%Der Grundgedanke des World Wide Web nach Tim Beners Lee war es ein offenes und freies Web zu schaffen um Wissen für die Allgemeinheit zugänglich zu machen. \todo{Quelle: Web-Report stelle suchen}
	Im folgenden wird kurz unterschieden zwischen den zwei Begriffen dezentral und förderiert.\\
	
	
	Protokolle wie \glqq Diaspora Federation und OStatus\grqq~bieten den Zugriff auf dezentral gespeicherte soziale Inhalte und ermöglichen das veröffentlichen von sozialen Inhalten. Der im März letzten Jahres vom \gls{w3c} empfohlene Standard namens \glqq ActivityPub\grqq~wurde auf Basis des Wissens im Umgang mit dem OStatus und Pump.io Protokoll entwickelt\cite{activityPub}.\\ 
	\todo{Quelle für dezentralen Zugriff}
	
	ActivityPub ist wie \glqq OStatus\grqq~ein Protokoll für dezentrale soziale Netzwerke und wird in dieser Bachelor Arbeit untersucht.
}

%\section{Unterschied zentraler sozialer zu dezentralen sozialen Netzwerken}
