%% LaTeX2e class for student theses
%% sections/content.tex
%%
%% Karlsruhe University of Applied Sciences
%% Faculty of  Computer Science and Business Information Systems
%% Distributed Systems (vsys)
%%
%% Prof. Dr. Christian Zirpins
%% christian.zirpins@hs-karlsruhe.de
%%
%%
%% Version 0.2, 2017-11-15
%%
%% --------------------------------------------------------
%% | Derived from sdqthesis by Erik Burger burger@kit.edu |
%% --------------------------------------------------------
\chapter{ 
	\iflanguage{english}{Introduction}{Einleitung}
}
\label{ch:Introduction}
\iflanguage{english}{
	\todo{- Justification of this work}
	Centralization of data and trust in individual instances is ubiquitous today. In the Internet there is a wealth of social networks such as Facebook, Twitter, Google+, Instagram or Pinterest which mostly retain the right to the data you get from their users and last but not least sell this data for advertising purposes. It is also easier for criminals and secret services to access the entire database, since centralised networks usually also have a central database or access point. When the access to this central point is reached, the whole database can be read.\\
	
	What does decentralized mean? In politics, decentralization \glqq is the transfer of central government tasks to subnational or subsidiary level(s)\grqq\cite{wikipedia-decentralization-politics}. In the energy industry, one speaks of \glqq decentralized power generation\grqq if the power is also generated at the places where it is consumed. An example of this would be a hydroelectric power plant that supplies electricity to the surrounding villages or cities\cite{wikipedia-decentralization-energy}. In computer science, decentralization means the distribution of data over several independent servers. \\
	
	By allocating tasks to sub-national or subsidiary levels, the central government level is relieved of its burden and thus has more resources for other tasks. When hydroelectric power plants are built close to villages and towns, there is no need to transport electricity over long distances, which reduces the loss of electricity when it is transported over the power grid. Decentralisation of social networks brings various advantages. One advantage is the scalability of decentralized social networks. By adding more instances, new resources can be made available to the network. Another advantage is that there is no central control body that can be corrupted by criminals, economic lobbyists or government authorities.\\
	
	% - Goal of the work
	The aim of the work is to implement the distributed server-to-server protocol as a prototype and to test the security standards for the ActivityPub protocol recommended by the \gls{w3c} community. In addition, the work should give an overview of the basics of the protocol and how the relevant components are to be understood. The security standards used relate to the authentication of the client against the server and additionally the server to each other, as well as ensuring the manipulation-free transmission of content.\\

}{
	%	- 	Rechtfertigung der Arbeit
	Zentralisierung von Daten und das Vertrauen auf einzelne Instanzen ist heutzutage allgegenwärtig. Im Internet gibt es eine Fülle an sozialen Netzwerken wie Facebook, Twitter, Google+, Instagram oder Pinterest die zumeist das Recht an den Daten, die Sie von ihren Nutzer bekommen, behalten und nicht zuletzt diese Daten auch verkaufen für z. B. Werbezwecke. Für Kriminelle sowie Geheimdienste ist es außerdem leichter an die gesamte Datenbank zu kommen, da bei zentralisierten Netzwerken meist auch eine zentrale Datenbank, bzw. ein zentraler Zugriffspunkt, vorhanden ist. Wenn der Zugriff auf diesen zentralen Punkt erreicht ist kann die ganze Datenbank ausgelesen werden.\\
	
	Was bedeutet nun dezentral? In der Politik wird unter Dezentralisierung \glqq die Übertragung zentral staatlicher Aufgaben auf subnationale oder subsidiäre Ebene(n) verstanden\grqq\cite{wikipedia-dezentralisierung-politik}. In der Energiewirtschaft spricht man von einer \glqq dezentralen Stromerzeugung\grqq, wenn der Strom an den Stellen wo er verbraucht auch erzeugt wird. Ein Beispiel hierfür wäre ein Wasserkraftwerk, dass den Strom für die umgebenen Dörfer oder Städte liefert\cite{wikipedia-dezentralisierung-energie}. In der Informatik versteht man unter Dezentralisierung das verteilen von u. a. Daten über mehrere unabhängige Server hinweg.\\
	
	Durch die Verteilung der Aufgaben auf sub-nationale oder subsidiäre Ebene wird die zentral staatliche Ebene entlastet und hat somit mehr Ressourcen für andere Aufgaben. Beim Errichten von Wasserkraftwerken nahe an Dörfern und Städten entfällt der Transport des Stroms über größere Distanzen und somit verringert sich der Verlust beim Transport über das Stromnetz. Dezentralisieren von sozialen Netzwerken bring verschiedene Vorteile mit sich. Ein Vorteil ist die Skalierbarkeit bei dezentralen sozialen Netzwerken. Durch das hinzufügen weiterer Instanzen können dem Netzwerk neue Ressourcen bereitgestellt werden. Ein weiterer Vorteil liegt darin, dass keine zentrale Kontrollinstanz vorhanden ist welche korrumpiert werden kann, sei es durch Kriminelle, wirtschaftliche Interessenvertreter oder staatliche Autoritäten.\\
	
	%	- 	Ziel der Arbeit
	Ziel der Arbeit ist es die Implementierung des verteilten Server-zu-Server Protokolls als Prototyp vorzunehmen und die bis dato von der \gls{w3c} Gemeinschaft empfohlenen Sicherheitsstandards für das ActivityPub Protokoll zu prüfen. Es soll die Frage geklärt werden ob HTTP- und Linked Data Signaturen bestmöglich dafür geeignet sind, die sichere Kommunikation zu gewährleisten anhand eines Vergleichs der beiden Verfahren. Darüber hinaus soll die Arbeit einen Überblick über die Grundlagen des Protokolls geben und wie die relevanten Bestandteile zu verstehen sind. In diesem Zusammenhang ist mit \glqq sicher\grqq~die authentische und manipulationsfreie Übertragung der Server untereinander gemeint.\\
	% Schutzziele
	%	- 	Abgrenzung des Themas und Themenbezogenen Definitionen
		
	%	- 	Geschichte und Stand der Forschung:
}

\section{Motivation}
\label{sec:Introduction:Motivation}
\iflanguage{english}{
	With central social networks, e.g. Facebook, the company has control over data. Due to decentralization, data remains with the individual instances of the social network. Although the databases in central networks can be distributed and the network can also be decentralized, the data still belongs to one company. In addition, central social networks cannot easily exchange content with other networks. With a protocol like ActivityPub, distributing and accessing content and sharing activities across multiple social networks is made possible.\\
	
	ActivityPub consists of two parts. The client-to-server protocol, also called \glqq Social API\grqq~, and the promoted server-to-server protocol. Through the ActivityPub \gls{w3c} recommendation in early 2018 and the \gls{swwg} working group of the \gls{w3c} and others, the dissemination of the protocol is promoted. \\
	
	The network effect is a term derived from economics. One example is the telephone network. The more people own a phone, the greater the benefit for each individual phone owner\cite{network effect}. Transferred to ActivityPub this means as much as: \glqq The more networks implement the standard, the greater the benefit for a single network and ultimately for individual users\grqq.\\
	
	%In the following, a brief distinction is made between the two terms decentralized and federated.\\
	
	Protocols such as \glqq Diaspora Federation and OStatus\grqq~provide access to decentralized social content and enable the publication of social content. The standard recommended in March last year by \gls{w3c} called \glqq ActivityPub\grqq~was developed on the basis of the knowledge in dealing with the OStatus and Pump.io protocol\cite{activityPub}.\\
	
	ActivityPub is like \glqq OStatus\grqq~a protocol for decentralized social networks and is examined in this Bachelor thesis.
}{
	Bei zentralen sozialen Netzwerken, z.B. Facebook, besitzt die Kontrolle über Daten das Unternehmen. Durch die Dezentralisierung bleiben Daten bei den einzelnen Instanzen des sozialen Netzwerkes. Die Datenbanken bei zentralen Netzwerken können zwar verteilt sein und auch das Netzwerk könnte dezentral angelegt sein, trotzdem gehören die Daten einem Unternehmen. Außerdem können bei zentralen sozialen Netzwerken nicht ohne weiteres Inhalte mit anderen Netzwerken ausgetauscht werden. Mit einem Protokoll wie ActivityPub wird das Verteilen sowie der Zugriff auf Inhalte und der Austausch von Aktivitäten über mehrere soziale Netzwerke hinweg ermöglicht.\\
	
	ActivityPub besteht aus zwei Teilen. Dem Client-zu-Server Protokoll, auch \glqq Social API\grqq~genannt, und dem förderierten Server-zu-Server Protokoll. Durch die ActivityPub \gls{w3c} Empfehlung Anfang 2018 und die \gls{swwg} Arbeitsgruppe des \gls{w3c} sowie weiteren, wird die Verbreitung des Protokolls vorangebracht.\\
	
	Der Netzwerkeffekt ist ein aus der Volkswirtschaftslehre stammender Begriff. Als Beispiel sei das Telefonnetz genannt. Umso mehr Leute ein Telefon besitzen, umso höher ist der Nutzen für jeden einzelnen Telefon Besitzer\cite{netzwerkeffekt}. Übertragen auf ActivityPub bedeutet das soviel wie: \glqq Je mehr Netzwerke den Standard implementieren, desto höher ist der Nutzen für ein einzelnes Netzwerk und im Endeffekt für die einzelnen Nutzer\grqq.\\
	%Der Grundgedanke des World Wide Web nach Tim Beners Lee war es ein offenes und freies Web zu schaffen um Wissen für die Allgemeinheit zugänglich zu machen. \todo{Quelle: Web-Report stelle suchen}
	%Im folgenden wird kurz unterschieden zwischen den zwei Begriffen dezentral und förderiert.\\
	
	
	Protokolle wie \glqq Diaspora Federation und OStatus\grqq~bieten den Zugriff auf dezentral gespeicherte soziale Inhalte und ermöglichen das veröffentlichen von sozialen Inhalten. Der im März letzten Jahres vom \gls{w3c} empfohlene Standard namens \glqq ActivityPub\grqq~wurde auf Basis des Wissens im Umgang mit dem OStatus und Pump.io Protokoll entwickelt\cite{activityPub}.\\ 
	\todo{Quelle für dezentralen Zugriff}
	
	ActivityPub ist wie \glqq OStatus\grqq~ein Protokoll für dezentrale soziale Netzwerke und wird in dieser Bachelor Arbeit untersucht.
}

%\section{Unterschied zentraler sozialer zu dezentralen sozialen Netzwerken}
