%% LaTeX2e class for student theses
%% sections/conclusion.tex
%%
%% Karlsruhe University of Applied Sciences
%% Faculty of  Computer Science and Business Information Systems
%% Distributed Systems (vsys)
%%
%% Prof. Dr. Christian Zirpins
%% christian.zirpins@hs-karlsruhe.de
%%
%%
%% Version 0.2, 2017-11-15
%%
%% --------------------------------------------------------
%% | Derived from sdqthesis by Erik Burger burger@kit.edu |
%% --------------------------------------------------------
\chapter{Fazit und Ausblick}
\todo{Im Fazit die Arbeit nochmal Zussammenfassen, blos kennt der Leser die Arbeit bereits}
\section{Fazit}
Ergebnisse, wie sind sie einzuordnen, 
\section{Ausblick}
\label{ch:Conclusion}
Da das Interface sehr groß werden kann beim hinzufügen weiterer Funktionalität könnte dieses neu Strukturiert oder aufgeteilt werden. Beispielsweise kann eine weitere Arbeit innerhalb des Interfaces Fassaden-Klassen anlegen. Es könnte sich auch ein neues Interface Design überlegt werden.\\

Die Implementierung beinhaltet zur Authentifizierung der Server untereinander sowie zur Sicherstellung der Datenintegrität Funktionalität zum Erstellen und Verifizieren von HTTP-Signaturen. Eine weitere Arbeit kann mit einer Implementierung der Funktionalität zum Signieren und Verifizieren von unterschiedlichen \textit{Linked Data} Signatur Typen und einer Performanz-Messung dieser Implementierung auf diese Arbeit aufbauen.\\

Zur Erweiterung der Funktionalität des Frameworks können weitere Werkzeugklassen sowie Logik zum Verarbeiten zusätzlicher Aktivitäten und Objekte hinzugefügt werden. Das Framework enthält bis dato keine seitens des Standards definierten Aktivitäten zum hinzufügen zu und entfernen von Aktivitäten aus Sammlungen (Add, Remove). Auch weitere Objekttypen wie \textit{Audio} oder \textit{Video} können hinzugefügt werden\footnote{siehe \url{https://www.w3.org/TR/activitystreams-vocabulary/}, 3.3}.\\

Auch eine Komponente zum Schutz vor förderierten \textit{denial-of-service} Attacken des förderierten Servers könnte implementiert werden. Der Nutzerinhalt von Objekten kann außerdem so bereinigt werden, dass kein \textit{cross-site-scripting} stattfinden kann.
